%
% File acl2014.tex
%
% Contact: koller@ling.uni-potsdam.de, yusuke@nii.ac.jp
%%
%% Based on the style files for ACL-2013, which were, in turn,
%% Based on the style files for ACL-2012, which were, in turn,
%% based on the style files for ACL-2011, which were, in turn, 
%% based on the style files for ACL-2010, which were, in turn, 
%% based on the style files for ACL-IJCNLP-2009, which were, in turn,
%% based on the style files for EACL-2009 and IJCNLP-2008...

%% Based on the style files for EACL 2006 by 
%%e.agirre@ehu.es or Sergi.Balari@uab.es
%% and that of ACL 08 by Joakim Nivre and Noah Smith

\documentclass[11pt]{article}
\usepackage{acl2014}
\usepackage{times}
\usepackage{url}
\usepackage{latexsym}
\usepackage{tabulary}

%\setlength\titlebox{5cm}

% You can expand the titlebox if you need extra space
% to show all the authors. Please do not make the titlebox
% smaller than 5cm (the original size); we will check this
% in the camera-ready version and ask you to change it back.


\title{Modeling the Fine-Grained Expression of Suicide-Related Ideas at Scale}

\author{Ravdeep Johar\\
  Department of Computer Science \\
  Rochester Institute of Technology \\
  {\tt rsj7209@rit.edu} \\\And
 Christopher M. Homan\\
  Department of Computer Science \\
  Rochester Institute of Technology \\
  {\tt cmh@cs.rit.edu}\\\And
  Second Author \\
  Affiliation / Address line 1 \\
  Affiliation / Address line 2 \\
  Affiliation / Address line 3 \\
  {\tt email@domain} \\}

\date{March 21, 2014}

\begin{document}
\maketitle
\begin{abstract}
 \end{abstract}

\section{Introduction}

Suicide is among the top ten leading causes of death in the United States, and when the age range is narrowed to include individuals 10-44 years in age, suicide is among the top five causes of death \cite{heron2009deaths}. Indeed, while mortality rates for most illnesses decreased between 2008 and 2009, the rate of suicide increased by 2.4\%  \cite{heron2009deaths}, and research suggests that the lifetime prevalence for suicidal ideation is between 5.6 and 14.3 percent, with an even greater prevalence among youth (19.8–24.0\%) \cite{nock2008suicide}. Approximately, one third of individuals who reported lifetime suicidal ideation made a plan and nearly three-quarters of those who reported having a suicide plan went on to attempt suicide \cite{kessler1999prevalence}. According to Kessler, Borges, and Walters  \cite{kessler1999prevalence}, the odds of attempting suicide increased exponentially when individuals endorsed three of more risk factors (e.g., having a mood or substance abused disorder).
	Although additional knowledge about risk factors for suicidal behaviors (i.e., suicidal ideation and suicide attempt) has clinical implications; traditional methods for researching suicidal behavior is influenced by sociocultural barriers such as stigma and shame as well as issues with methodology \cite{crosby2011self}. For instance, data is often collected from healthcare organizations or large-scale studies that are often limited depending on providers’ ability to assess and document suicide risk or from self-report measures \cite{crosby2011self,horowitz2009suicide}. However, computational linguistics methods informed by surveillance techniques as well as a theoretical model of suicidal behavior may provide additional insight into this significant public health concern. Especially since individuals may be more inclined to seek out support from informal resources such as social media networks instead of seeking treatment \cite{crosby2011self,bruffaerts2011treatment,ryan2010universal}. Evidence suggests that youth and emerging adults usually prefer to seek help from their friends and families; however, higher levels of suicidal ideation had an inverse relationship with all types of help-seeking and a positive correlation with the decision to not seek support \cite{deane2001suicidal}.   
	Aside from demographic variables, previous suicide attempts, mental health concerns (i.e., depression, substance abuse, suicidal ideation, self-harm, and impulsivity), family history of suicide, interpersonal conflicts (i.e., family violence and bullying), mechanism or means for suicidal behavior (e.g., firearms)  are commonly cited risk factors for suicidal behavior \cite{nock2008suicide,crosby2011self,gaynes2004screening,harriss2005suicidal,shaffer2004columbia,shaffer2004columbia,brown2000risk}. Suicide is a complex phenomenon with a low base-rate; therefore, many researchers tend to focus on examining the relationship between risk factors and suicidal behavior without relying heavily on theoretical models \cite{nock2008suicide} .  However, Mann, Waternaux, Haas, and Malone \cite{mann1999toward}, developed the stress-diathesis model for suicidal behavior using many of the aforementioned risk factors. Specifically, this framework suggests that objective states such as depression and life events as well as subjective states and traits such as a family history of depression and/or suicide as well as substance abuse were among the risk factors that contributed to suicidal ideation and could eventually lead to either externalizing (e.g., interpersonal violence) or internalizing aggression (e.g., attempting suicide) \cite{mann1999toward}.
	Since the stress-diathesis model was developed using risk factors for suicidal behavior, it is the ideal framework to analyze publically available linguistic data from social media outlets such as Twitter. While traditional methods are limited by survey questions or clinical reports, data from social media can be used as a natural experiment to examine depression and suicidal ideation without being constrained by such sample biases as individuals who are willing take part in research and/or seek out formal sources of support. Moreover, this natural experiment method may provide information about individuals who are unlikely to engage in formal help-seeking behaviors and eventually it could be used to identify effective methods in natural helping. Hence, this universal approach to screening for suicidal behaviors may have future implications not only for identifying individuals who have a higher prevalence for suicidal behaviors but it could eventually lead to the methods for enhancing protective factors against suicide.  
 
\textbf{[CMH: Vince or Megan, please elaborate. I'm looking for terse, citable facts about suicide]} Suicide is epidemic in some populations. It is the th leading cause of death. In nearly all populations, it is on the rise.

\textbf{[CMH: Vince or Megan, I'm making a lot of claims about clinical practice complete ad hoc. Please rewrite as necessary.]} From a clinical perspective, suicide is a social-behavioral problem one that is related to other social-behavioral problems, such as depression or drug addiction. Unfortunately, due to a number of reasons, including social stigma, individuals afflicted with such pre-suicidal conditions can often not be replied upon to report their behavior.  Suicidal behavior is usually not discovered until the victim acts, and even then diagnosis is muddy. 

Suicide is extremely infrequent and usually the result of psychological distress that takes several weeks at the very least to manifest. How can social media and its primary virtues, namely high sample rates and volume [maybe a reference here to that Alex Pentland diagram, Vince?], help to detect suicidality? Certainly, one would expect that within the sheer volume of data that the most popular social media services produce must lie some evidence of suicide ideation. Suicidal behavior, however, is extremely difficult to diagnose even in the best of situations. Rather than pick out obvious rare cases, our goal here is to detect in the language of microblog users expressions 
of behavior or ideas related to suicide. Most of this language will, of course, not indicate actual suicide ideation... [something about painting a picture of how factors related to suicide lie in a virtual social space.]

In contrast to the way that suicide is detecting in clinical settings [Megan, hopefully you will say something about this in the clinical subsection above], the high sample rates that social media provide allow us to detect clues of suicide-related behavior as they occur, without relying on self reporting by often highly unreliable sources. 

observe behavior
in a natural setting, when the 

\subsection{Research Questions}
\begin{description}
\item[R1] Using support vector machines trained on primarily linguistic features, can we learn to classify tweets as whether or not they express ideas related to suicide?
\item[R2] Can network structure enhance such classification models?
\item[R3] Can we predict the expression of suicide-related ideas in the future?
\end{description}
\subsection{Related Work}
\subsubsection{Clinical [Megan]}
\subsubsection{Computational-Linguistic [Cissi]}
\subsubsection{Affect in Social Media [Chris]}

Sentiment analysis has been widely studied in a number of computational settings, including on various social networking sites. Relatively little of this work has focused on suicide or related psychological conditions. \cite{masuda2013suicide} study suicide on mixi. \cite{cheng2012opportunities} consider the ethical and political implications of online data collection for suicide prevention. \cite{Jay} show correlations between frequency in tweets related to suicide and actual suicide in the 50 United States of America.

A rather substantial body of work already exists on the use of Twitter to study emotion~\cite{bollen2011twitter,dodds2011temporal,wang2012harnessing,pfitzner2012emotional,kim2012you,bollen2011happiness,pfitzner2012emotional,bollen2011modeling,mohammad2012emotional,golder2011diurnal,de2012not,de2012happy,de2013major,de2013understanding,hannak2012tweetin,thelwall2011sentiment,pak2010twitter}, though comparatively little of it considers network properties. For instance,
Golder and and Macy study aggregate global trends in ``mood,'' and show, for example, that people wake up in a relatively good mood that decays as the day progresses \cite{golder2011diurnal}, Bollen et al.~\cite{bollen2011modeling} show that POMS-scored tweets are often tied to current events, such as elections and holidays.
 Bliss et al., building on~\cite{dodds2011temporal} study assortativity with respect to happiness on Twitter~\cite{bliss2012twitter}. Coviello et al. study the spread of mood in Twitter~\cite{coviello2014}. They notice a very small---but statistically significant---spreading of mood over Facebook. Sadilek et al. study depression on Twitter~\cite{sadilek2014modeling}. De Choudhury studied depression in general, and in particular in post-partem, in Twitter~\cite{de2012not,de2012happy,de2013major,de2013understanding} and Facebook. Homan et al. investigate depression in TrevorSpace~\cite{homan2014social}. 

Much research on social networking sites focuses on diffusion, or the idea that behaviors, attitudes, or even emotions can spread from person to person through social influence, like a disease. This approach has often been criticized as too reductive. Also, it is difficult to distinguish influence from selection.

Social networking services, such as Facebook, hold great promise in many social disciplines as a means to test theories relating social environment to any number of psychological problems. Regarding suicide alone, there a number of such theories \cite{wray2011sociology}.  A widely-used~\cite{coviello2014,bollen2011happiness}, basic measurement for testing such theories is assortativity, or the Pearson correlation coefficient~\cite{newman2002assortative}.
Major theories. Influence vs. selection. Social support. Read criticisms. Cultural Norms. Eric's paper.
Centola, refs in christakis and fowler paper, criticism of c\&f, relationship between emotion and social network usage. more suicide based research. Thompson.

Another major theme in social network analysis is the idea that ties represent different kinds of relationships. At the most basic level, one can distiguish between \emph{weak} and \emph{strong} social ties and observe different behavior and effects between them.  

Following in the work of Bliss et al.~\cite{coviello2014} and Bollen et al.~\cite{bollen2011happiness} we show that 
mood is assortative. We additionally consider the predictive power of various measureable notions of tie strength.
We study suicide risk factors here but we would expect our methods would apply to other domains.



\section{Methods}
\subsection{Overview}
\textbf{CMH: Mention coding as the method of obtaining ground truth; provide justification for it.}
\subsection{Data}
Twitter is a worldwide popular social networking and microblogging service. Twitter message contains up to 140 words each and such words limit encourages users to update frequently. User's regular posts on Twitter have been used to predict depression[Munmun De Choudhury et al., 2013], influenza-like illnesses[Adam Sadilek et al., 2012] in previous studies. Our research looked into an old Twitter dataset originated from the New York City, which covered a month long period since May 18, 2010, total with about 2.5 million tweets from 6,237 unique users. See Table~\ref{table::dataset}.

\begin{table}[t]
\small
\centering
\begin{tabular}{ l | r }
\multicolumn{2}{c}{\textbf{New York City Dataset}} \\
\hline
\textbf{Unique users}&       632,611      \\
\hline
\textbf{Unique geo-active users}&        6,237       \\
\hline
\textbf{Tweets total}&        15,944,084\\
\hline
\textbf{GPS-tagged tweets}&   4,405,961\\
\hline
\textbf{GPS-tagged tweets by geo-active users}&  2,535,706\\
\hline
\textbf{GPS-tagged tweets by geo-active users}&  2,047 \\
\textbf{that show a symptom of an illness}&\\
\hline
\textbf{``Follows'' relationships}&  102,739  \\
\textbf{between geo-active users}&\\
\hline
\textbf{``Friends'' relationships}&   31,874   \\
\textbf{between geo-active users}&\\
\end{tabular}
\caption{\small Summary statistics of the data collected from NYC. Geo-active users are people who geo-tag their tweets relatively frequently (more than 100 times per month). Note that the reciprocity rate in the social graph is about 31\%, which is consistent with previous findings  cite what is twitter.} 
\label{table::dataset}
\end{table}
\subsection{Data Collection}
-slang replace
-strip punctuations

\textbf{CMH: \cite{Jay}. It is important to give them credit for what they did, and to note that the scheme you describe was adopted from them almost verbatim.}

To identify suspected suicide tweets, we created a list of inclusive search terms/phrases according to various risk factors and warning signs linked to suicide. [Table 1]
\begin{table*}
    \centering
    \begin{tabulary}{500pt}{|C|C|}
    \hline
    \textbf{Suicide Risk Factor Category} & \textbf{Search Terms and Phrases} \\ \hline
    Depressive Feelings & ~  me abused depressed, tired of living,so depressed,leave this world, wanna die,me hurt depressed, feel hopeless depressed, feel alone depressed, i feel helpless, i feel worthless, i feel sad, i feel empty, i feel anxious, hate my job, feeling guilty, deserve to die, desire to end own life, feeling ignored, tired of everything, feeling blue, have blues                                                                                                                                                                                                                                                                                                                                                                                                                                                                         \\ \hline
    Depression Symptoms                   & sleeping pill, sleeping a lot, i feel irritable, i feel restless, have insomnia, sleep forever,sleep disorder                                                                                                                                                                                                                                                                                                                                                                                                                                                                                                                                                                                                                                                                                                                             \\ \hline
   Drug Abuse                            & depressed alcohol, sertraline, zoloft, prozac, pills depressed, clonazepam, drug overdose, imipramine                                                                                                                                                                                                                                                                                                                                                                                                                                                                                                                                                                                                                                                                                                                                     \\ \hline
  Prior Suicide Attempts                & suicide once more, me abused suicide, pain suicide, tried suicide                                                                                                                                                                                                                                                                                                                                                                                                                                                                                                                                                                                                                                                                                                                                                                         \\ \hline
  Suicide Around Individual             & mom suicide tried, sister suicide tried, brother suicide tried, friend suicide, suicide attempted, suicide attempt                                                                                                                                                                                                                                                                                                                                                                                                                                                                                                                                                                                                                                                                                                                        \\ \hline
   Suicide Ideation                      & commit suicide,committing suicide,feeling suicidal, suicide thought about, thoughts suicide, think suicide, thought killing myself, used thought suicide, once thought suicide, past thought suicide, multiple thought suicide, want to suicide, shoot myself, a gun to head, hang myself, intention to die                                                                                                                                                                                                                                                                                                                                                                                                                                                                                                                               \\ \hline
   Self Harm                             & stop cutting myself, hurt myself, cut myself                                                                                                                                                                                                                                                                                                                                                                                                                                                                                                                                                                                                                                                                                                                                                                                              \\ \hline
   Bullying                              & i am being bullied, i have been cyber bullied, was bullied, feel bullied, stop bullying me, keeps bullying me, always getting bullied                                                                                                                                                                                                                                                                                                                                                                                                                                                                                                                                                                                                                                                                                                     \\ \hline
   Gun Ownership                         & gun suicide, shooting range went, gun range my                                                                                                                                                                                                                                                                                                                                                                                                                                                                                                                                                                                                                                                                                                                                                                                            \\ \hline
   Psychological Disorders               & diagnosed schizophrenia, diagnosed anorexia, diagnosed bulimia, i diagnosed ocd, i diagnosed bipolar, i diagnosed ptsd, diagnosed borderline personality disorder, diagnosed panic disorder, diagnosed social anxiety disorder, diagnosed post traumatic stress disorder, sleep apnea                                                                                                                                                                                                                                                                                                                                                                                                                                                                                                                                                     \\ \hline
   Family Violence Discord               & dad fight again, parents fight again, lost my friend, argument with wife, argument with husband, shouted at each other                                                                                                                                                                                                                                                                                                                                                                                                                                                                                                                                                                                                                                                                                                                    \\ \hline
   Impulsivity                           & i impulsive, i am impulsive                                                                                                                                                                                                                                                                                                                                                                                                                                                                                                                                                                                                                                                                                                                                                                                                               \\ \hline
   Sad                                   & abandon, ache, aching, agoniz, agony, alone, broke,cried, cries, crushed, cry, crying, damag, defeat, depress, depriv, despair, devastat, disadvantage, disappoint, discourag, dishearten, disillusion, dissatisf, doom, dull, empt, fail, fatigu, flunk, gloom, grave, grief, griev, grim, heartbreak, heartbroke, helpless, homesick, hopeless, hurt, inadequa, inferior, isolat, lame, lone, longing, lose, loser, loses, losing, loss, lostlow, melanchol, miser, miss, missed, misses, missing, mourn, neglectoverwhelm, pathetic, pessimis, piti, pity , regret, reject, remorse, resign, ruin, sad, sadde, sadly, sadness, sob, sobbed, sobbing, sobs, solemn, sorrow, suffer, suffered, sufferer, suffering, tears, traged, tragic , unhapp,unimportant, unsuccessful, useless, weep, wept, whine, whining, woe, worthless, yearn \\ \hline

  
    \end{tabulary}
\caption{Search phrases for categories}
  \label{Table:1}
\end{table*}


Among these, terms in Sad category were generated from LIWC [http://www.liwc.net/index.php]. All the rest were concluded from depression and other psychological disorders (Lewinsohn, Rohde, \& Seely, 1994), prior suicide attempts (Lewinsohn et al., 1994), family violence, family history of drug abuse, firearms in the home, and exposure to the suicidal behavior of others (National Insti- tute of Mental Health, 2012). Other search terms included common antidepressants, as well as phrases that indicated suicide (Hawton, Zahl, \& Weaterall, 2003), ideation (American Foundation for Suicide Prevention, 2012a), deliberate self-harm (Zahl \& Hawton, 2004), bullying (Klomek, Sourander, \& Gould, 2011), feelings of isolation (CDC, 2012), and impulsiveness (American Foundation for Suicide Prevention, 2012b). [Self-directed Violence Surveillance: Uniform Definitions and Recommended Data Elements, report from Megan]

Before searching, we did some kinds of preprocess work:  (1) converted all text to lower case (2) stripped out all the punctuations and special characters; (3) built a slang dictionary which contains 5424 text slang based on online resources [http://www.noslang.com/dictionary/], Internet slang, and abbreviations. We replaced all the matching items fount in the dataset with corresponding easy read words. These two steps helped us extract more suicide-related tweets for analysis in the following process.


\subsection{Preprocess}
-stopwords
-keep stemming words

Using Scikit-learn package(http://scikit-learn.org/stable/about.html\#citing-scikit-learn), it is very simple to remove stop words (like ``and'', ``the'', etc.) with parameter ``stop\_words'' tuned as ``string \{`english'\}''. 

At the same time, we kept original words without any word stemming, i.e.: ``computers'', ``computing'', and ``compute'', neither of these words will be mapped to the same stem ``comput''. The reason here is: a tweet containing like ``soooooooooo sad!" is supposed to convey stronger emotions than simple ``so sad''.

\subsection{Ground Truth}
\textbf{CMH: Explain the annotation procedure here.}

 
\subsection{Analysis}
-topic modeling \cite{Blei}

\subsection{features}
-unigram, bigram, trigram
-tf-idf
As features, we used unigram, bigram and trigram tokens in the whole training data as the attributes. For example, a simple tweet ``I am so depressed'' is represented as the following feature vector: \{I, am, so, depressed, I am, am so, so depressed, I am so, am so depressed\}. The tf-idf values were calculated for each attribute: tf-idf stands for ``term frequency - inverse document frequency'', which is a numerical statistic to reflect how important a tokenization is to a document in a collection or corpus. The tf-idf value increases proportionally to the number of times a tokenization appears in the dataset, but is offset by the frequency of the tokenization in the corpus, which helps to control for the fact that some words are generally more common than others. With the help of Scikit-learn package(http://scikit-learn.org/stable/about.html\#citing-scikit-learn), there values can be easily acquired. 

\subsection{feature selection}
chi2 feature selection

\subsection{Annotation}
-expert VS non-expert


\subsection{SVM part}
-scikit learn sth....

\subsection{Network Features}
One of the fundamental properties of social networks appears to be \emph{tie strength}, or how ``close'' socially two people are. A large body of literature suggests that people are more likely to share personal information with stronger ties, and that weak ties play an important role in providing new information.  
Measuring tie strength is problematic, as there is no gold standard here. Social networking services seem to exacerbate the disparity between strong and weak ties, as many have ``friends'' or ``followers'' whom they may not even know personally, and also create their own problems and opportunities for estimating tie strength. In large-scale network analysis, researchers have sometimes characterized tie strength by the 
\emph{embeddedness} of an edge, which is the number of friends in common that two actors sharing a tie have. Highly embedded links are part of a strong social fabric, and represent strong ties. Another method of estimating tie strength is to measure the amount of activity between users. In this study, we investigate both the role that embeddness and activity play in corrolating suicide-related language. 

\section{Results}
\subsection{Strength of Ties}
Table~\ref{tab:strength} shows how the use of sad language (measure by the LIWC sad feature) correlates among users sharing different tie strength and between personal and broadcast messages.

\begin{table}
  \centering
  \begin{tabular}{l|r|r}
Embeddedness    &Broadcast&Personal\\
\hline
0 & 0.046 & 0.098\\
\hline
1 & 0.061 & 0.13\\
\hline
5& 0.082 & 0.23
  \end{tabular}
\label{tab:strength}
\caption{Pearson correlation coefficient of LIWC sad (aggegrated over all tweets collected) of broadcast and reciprocated personal tweets.}
\end{table}
\section{Discussion}
\subsection{Limitations}
As ground truth, we rely on tweets hand-annotated by experts and non-experts. However, the mental state of another individual, observed from a line or two of often slang-ridden \textbf{[yellow card! Cissi if you could replace this with your preferred term, that would be great]} text is necessarily hard to discern and, even under less noisy conditions, extremely subjective; even the observers' personal understandings of such concepts as ``distress'' may differ drastically. This makes inter-annotator agreement quite a challenge, to say nothing of observation in some objective fashion of the true mental state.

We hope we have shown here that even under such conditions, the sheer volume of data all but guarantees that some troubled individuals can be rather easily discovered.
 
\section{Conclusion and Future Work}
\section*{Acknowledgments}


% include your own bib file like this:

\bibliographystyle{acl}
\bibliography{acl2014}

%  remember to run:
% latex bibtex latex latex
% to resolve all references

\end{document}
